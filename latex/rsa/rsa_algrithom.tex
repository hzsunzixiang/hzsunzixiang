
\documentclass{article}
\usepackage{blindtext}
%\usepackage[utf8]{inputenc}

\usepackage{CJK}         % CJK 中文支持
 

% 不知为何 title显示中文 必须采用这种方式 否则报错 而document中是可以的 {CJK}{UTF8}{gbsn} 这种方式的
\begin{CJK*}{UTF8}{gbsn}


\title{\huge{教你透彻理解RSA公钥加密算法} \thanks{}}
%\title{\huge{教你透彻理解RSA公钥加密算法} }
\author{孙自翔\\[2pt]
\normalsize
(联系方式:haichengsun123@163.com) \\[2pt]}
\date{}  % 这一行用来去掉默认的日期显示



\begin{document}

% 注意这里 必须在摘要之前
\maketitle

%{\small{\textbf{一些说明}\quad The peaches in the birthday party of lady Wang Mu were so delicious that I want to dwell on the analysis and simulation on them. So that I can bring some of them to my kids in Hua Guo Shan.\\
%\textbf{Key Words}\quad Peach, lady Wang Mu, birthday party, Heaven palace}}
% 
一些说明: 本文主要讨论RSA算法所涉及的每个数学方面的知识点,主要是数论方面。采用总分总结构:首先简要介绍RSA加密算法,然后分述各知识点,最后以一个实际例子结束。
 
 
\section{RSA算法介绍}
 
This is the first section.
 
Lorem  ipsum  dolor  sit  amet,  consectetuer  adipiscing  
elit.   Etiam  lobortisfacilisis sem.  Nullam nec mi et 
neque pharetra sollicitudin.  Praesent imperdietmi nec ante. 
Donec ullamcorper, felis non sodales...
 
\section{数论方面的知识点}
\subsection{同余}   
  %https://zh.wikipedia.org/wiki/%E5%90%8C%E9%A4%98  
  %\\ 叙述:
\subsection{裴蜀等式(Bézout's identity)}   
\subsection{辗转相除法}   
\subsection{扩展欧几里得算法}   
\subsection{模逆元(模反元素)}   
\subsection{欧拉定理}   
\subsection{欧拉函数}   
\subsection{费马小定理}   
\subsection{RSA算法}   
 
Lorem ipsum dolor sit amet, consectetuer adipiscing elit.  
Etiam lobortis facilisissem.  Nullam nec mi et neque pharetra 
sollicitudin.  Praesent imperdiet mi necante...

  本文叫你透彻理解RSA公钥加密算法 
The Triangulation of Titling Data in Non-Linear Gaussian Fashion via $\rho$ Series\thanks No procrastination
这是一个换行\\
换行咯
\section{实例说明}

\end{CJK*}
 
\end{document}
