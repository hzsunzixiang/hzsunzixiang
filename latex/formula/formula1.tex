
\documentclass{article}
\usepackage{CJKutf8}
 
\begin{document}

\begin{CJK}{UTF8}{gbsn}
  Hello World!
  可以输入中文
  %符号表  https://reu.dimacs.rutgers.edu/Symbols.pdf
  符号表  https://reu.dimacs.rutgers.edu/Symbols.pdf
  $\equiv$

  $\alpha$

  $  \log|x||y|b $

  $  \log\left|x\right|\left|y\right|b $

  % https://tex.stackexchange.com/questions/43008/absolute-value-symbols
  $  \log\mathopen|x\mathclose|\mathopen|y\mathclose|b $
\end{CJK}


% http://blog.csdn.net/bensnake/article/details/43279329
\begin{CJK}{UTF8}{gkai}  
这是一个楷体中文测试,处理简体字。  
\end{CJK}  
\begin{CJK}{UTF8}{gbsn}  
这是一个宋体中文测试,处理简体字。  
\end{CJK}  
\begin{CJK}{UTF8}{bkai}  
這是一個big5編碼的楷體中文測試,處理繁體文字。  
\end{CJK}  
\begin{CJK}{UTF8}{bsmi}  
這是一個个big5編碼的明體中文測試,處理繁體文字。  
\end{CJK}  

${-b\pm\sqrt{b^2-4ac} \over {2a}}$

\({-b\pm\sqrt{b^2-4ac} \over {2a}}\)


I can write math equation in line with text, for example: $E=mc^2$. Just wrap the equation with two dollar sign.

Or we can use \(E=mc^2\), or \begin{math}E=mc^2\end{math}       % 除了使用$...$美元符号,还可以使用命令\(...\)或者math环境\begin{math}...\end{math}

\section{Displayed formula}
The displayed formula is not in line and get numbered, I can use the equation environment:      % 单独占据整行居中展示出来的,称为显示数学公式,或行间公式、列表公式
\begin{equation}
-\frac{\hbar^2}{2m}\frac{d^2\Psi}{dx^2}=E\Psi       % 这种方式带编号,下面另外几种方式不带编号
\end{equation}



Or I can use four dollar sign: $$E=mc^2$$       % 或者使用\[...\]命令,以及displaymath环境

Or we can use
\[
E=mc^2      % 为了使代码更清晰,建议把公式单独放一行,也可以\[E=mc^2\]
\]

Or
\begin{displaymath}
E=mc^2
\end{displaymath}


\section{Symbols}
\subsection{Alphabet and normal symbol}
The default math fonts use mathnormal:

the Latin alphabet:     % 拉丁字符
$$a b c d e f g h i j k l m n o p q r s t u v w x y z 1 2 3 4 5 6 7 8 9 0$$

$$\mathnormal{x y z}$$

the Greek alphabet:     % 希腊字符
\[
\alpha \beta \gamma \delta \eta \theta \lambda \mu \pi \sigma \phi \tau \omega \psi \phi
\]

Here is a list of frequently used math symbols: 

$\angle A = \pi / 2$, 
$\pi = 180^\circ$,      % latex默认的数学字体中没有专用于表示角度的符号,自然也没有这个命令,所以通过上标输入
$f \circ g$,        % \circ是一个通常用来表示函数复合的二元运算符
$x+y$,      % 即使是简单的x+y,也应该用数学模式

compared with normal text: x+y  ,   % 在数学模式下,符号会使用单独的字体,字母通常是倾斜的意大利体,数字和符号则是直立体。甚至数学符号之间的距离也与一般的水平模式不同。

$\mathrm{e}^{\pi\mathrm{i}} + 1 = 0$

\subsection{Math operator}
\[
\int f(x) \, \mathrm{d} x       % 注意积分式的写法,微分算子d应该使用直立罗马体,后面的变量仍是默认的意大利体,并且用\.与前面的被积函数分开
\]

\[
\cos 2x = \cos(x+x) = \cos^2 x - \sin^2 x
\]


\section{Structures}
Math formula is not just about symbols, it has structures.
\subsection{Superscript and Subscript}
$A_{ij}=2^{i+j}$, $v_0 = v_1 + v_2$, $c^2 = a^2 + b^2$      % 当上标下标多于一个字符时,需要使用分组确定上下标范围

$$A_i^k = B^k_i$$       % 上下标同时使用,互不影响

$$K_{n_i} = K_{2^i} = 2^{n_i} = 2^{2^i}$$       % 嵌套使用上下标时,外层一定要使用分组

\[
a = a', b_0' = b_0'', {c'}^2 = (c')^2       % 撇号是一种特殊的上标,可以与下标混用,可以连续使用(普通的上标不能连续使用),但不能与上标直接混用
\]

\[
\max_n f(n) = \sum_{i=0}^n A_i          % 在显示公式中,多数数学算子的上下标是在正上或正下方
\]

$\max_n f(n) = \sum_{i=0}^n A_i$        % 注意在行内公式(inline formula)中,为了避免过于拥挤或产生难看的行距,所有算子的上下标也都在角标的位置了

\subsection{Fraction}
\[
\frac {1}{2} + \frac {1}{a} = \frac{2+a}{2a}
\]

$\frac {1}{2} + \frac {1}{a} = \frac {2+a}{2a}$     % 在行内公式和显示公式中,分式的大小是不同的

\subsection{Other Structure}
braces, squares, matrix, etc.

\section{Examples}
The lateral resolution of a microscope:
%$$r_{lat} = \frac {0.61 \lambda}{\text{NA}}$$
%%
%%The axial resolution of a microscope:
%%$$r_{axi} = \frac {2 \lambda \eta}{(\text{NA})^2}$$

\end{document}






