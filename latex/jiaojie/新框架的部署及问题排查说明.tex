
\documentclass{article}
\usepackage{CJKutf8}
\usepackage{mathrsfs}
\usepackage{titlesec}
\usepackage{helvet}

\usepackage{subcaption}
\usepackage{setspace}


% 如何设置页眉 还没有学会
%\usepackage{fancyhdr}                                
%\usepackage{lastpage}                                           
%\usepackage{layout}                                             
%\footskip = 10pt                                                
%%\pagestyle{fancy}                    % 设置页眉                 
%\chead{boyaa}

\setcounter{tocdepth}{3} 
\doublespacing

\usepackage{amssymb}% http://ctan.org/pkg/amssymb 特殊符号






% 注意用gbsn 的 否则 开头有一个中文无法显示出来
%\begin{CJK*}{UTF8}{gbsn}
\begin{CJK*}{UTF8}{gkai}

%\title{\huge{透彻理解RSA公钥加密算法} \thanks{}}
\title{\huge{德州Server新框架的部署及问题排查说明} }
\author{孙自翔\\[2pt]
(Email:haichengsun123@163.com) \\[2pt]}
\date{}  % 这一行用来去掉默认的日期显示


%http://latex.org/forum/viewtopic.php?t=13545

\usepackage{amsthm}
\newtheorem{thm}{Theorem}[section]
\theoremstyle{definition}
\newtheorem{dfn}{定义}[section]
\theoremstyle{remark}
\newtheorem{note}{Note}[section]
\theoremstyle{plain}
\newtheorem{lem}[thm]{Lemma}


 
\begin{document}

% 注意这里 必须在摘要之前
\maketitle  % 这里和页眉冲突


%2. pdflatex 运行两次失败,原因是sections标题中有中文  通过删除 rsa_alg.toc 来解决
%	但是生成tableofcontents又需要两次 生成toc
%	和第三条冲突了
%
%3. 用pdflatex 生成tableofcontents 需要运行两次才行

%所以暂时不生成了  需要再次研究
%\tableofcontents
%\addtocontents{toc}{\setcounter{tocdepth}{3}} % Set depth to 1
%\newpage

说明: 本文从数学的角度,阐述RSA算法所涉及的数论方面的各个知识点,文章首先简要介绍RSA加密算法,然后分述各数学知识点,最后以一个实际例子结束。

% 
% 这里有中文字体警告 
\section{密码学术语介绍}
%$\mathscr{ABCDEFGHIJKLMNOPQRSTUVWXYZ}$ % 花体必须大写 小写不行
% 
\hspace*{6mm}
概括地讲\cite{BOOK:1},密码系统是指包含可能的明文信息的有限集合 $\mathscr{P}$,可能的密文信息的有限集合$\mathscr{L}$,可能的密钥的密钥空间$\mathscr{K}$,以及对于密钥空间$\mathscr{K}$里的每一个密钥$k$,存在加密函数$E_k$和对应的解密函数$D_k$,使得任意的明文信息$x$满足$D_k(E_k(x))=x$。


%\section{\textbf{数论概念及定理阐述}}
\section{数论概念及定理阐述}
\subsection{整除}   

一个整数可以被另一个整数整除的概念在数论中处于中心地位.
\begin{dfn}
如果$a$和$b$为整数且$a \neq 0$, 我们说$a$整除$b$是指存在整数$c$使得$b=ac$.如果$a$整除$b$,我们还称$a$是$b$的一个因子,且称$b$是$a$的倍数.
\end{dfn}
如果$a$整除$b$,则将其记为$a \mid b$,如果$a$不能整除$b$,则将其记为 $a \nmid b$

比如:$13 \mid 182$,$-5\mid30$,$6 \nmid 44$,$7 \nmid 50$

\subsection{模除}   
\begin{dfn}
模除\cite{WEBSITE:modulo}是一种不具交换性的二元运算。在C语言中用 $\%$表示。
\end{dfn}

%\paragraph
当 $a = bq + r$, q是整数,并使其达到最大,此时我们说a模除b等于r。(r为非负数)。

以数学式子表示:a模除b = $a-\left\lfloor \frac{a}{b}\right\rfloor \times b$。

例如要计算100模除16,由于$100/16$是一个大于6且不大于7的数,取q=6。结果为$100-16\times6=4$。



\subsection{同余}   
同余的语言使得人们能用类似于处理等式的方式来处理整数关系。

\begin{dfn}
设m是正整数. 若$a$和$b$是整数,且 $m \mid (a-b)$
\end{dfn}

同余\cite{WEBSITE:congruence_modulo}(英语:congruence modulo,符号:$\equiv$)是数论中的一种等价关系。当两个整数除以同一个正整数,若得相同余数,则二整数同余。同余是抽象代数中的同余关系的原型[3]。最先引用同余的概念与“≡”符号者为德国数学家高斯。


\subsection{辗转相除法}   
\subsection{扩展欧几里得算法}   
\subsection{模逆元(模反元素)}   
\subsection{欧拉定理}   
\subsection{欧拉函数}   
\subsection{费马小定理}   
\subsection{RSA算法}   
 

% 位于document中
\bibliography{rsa} 
\bibliographystyle{ieeetr}

\end{CJK*}

\end{document}
