

% UTF-8 encoding
\documentclass{article}
\usepackage{xeCJK}
\usepackage{mathrsfs}
\usepackage{titlesec}
\usepackage{helvet}

\usepackage{subcaption}
\usepackage{setspace}
\setcounter{tocdepth}{3} 
\doublespacing

\setCJKmainfont{FangSong}

%\title{\huge{透彻理解RSA公钥加密算法} \thanks{}}
\title{\huge{透彻理解RSA公钥加密算法} }
\author{孙自翔\\[2pt]
(Email:haichengsun123@163.com) \\[2pt]}
\date{}  % 这一行用来去掉默认的日期显示



%http://latex.org/forum/viewtopic.php?t=13545
\titleformat{\chapter}[display]
   {\normalfont\fontsize{16pt}{18pt}\selectfont\mdseries}
   {\MakeUppercase{\chaptertitlename}\ \thechapter}{20pt}{\MakeUppercase}
\titleformat{\section}
   {\normalfont\fontsize{12pt}{14pt}\selectfont\bfseries}{\thesection}{1em}{}
\titleformat{\subsection}
   {\normalfont\fontsize{12pt}{14pt}\selectfont\itshape}{\thesubsection}{1em}{}
\titlespacing*{\section} {0pt}{0pt}{2.3ex plus .2ex}
\titlespacing*{\subsection}{0pt}{0pt}{1.5ex plus .2ex}

\begin{document}

% 注意这里 必须在摘要之前
\maketitle

\tableofcontents
\addtocontents{toc}{\setcounter{tocdepth}{3}} % Set depth to 1
%\singleplacing

\newpage



说明: 本文从数学的角度,阐述RSA算法所涉及的数论方面的各个知识点,文章首先简要介绍RSA加密算法,然后分述各数学知识点,最后以一个实际例子结束。
% 
% 这里有中文字体警告 
\section{密码学术语介绍}
%$\mathscr{ABCDEFGHIJKLMNOPQRSTUVWXYZ}$ % 花体必须大写 小写不行
% 
\hspace*{6mm}
概括地讲\cite{BOOK:1},密码系统是指包含可能的明文信息的有限集合 $\mathscr{P}$,可能的密文信息的有限集合$\mathscr{L}$,可能的密钥的密钥空间$\mathscr{K}$,以及对于密钥空间$\mathscr{K}$里的每一个密钥$k$,存在加密函数$E_k$和对应的解密函数$D_k$,使得任意的明文信息$x$满足$D_k(E_k(x))=x$。


%\section{\textbf{数论概念及定理阐述}}
\section{数论概念及定理阐述}
\subsection{同余}   
  %https://zh.wikipedia.org/wiki/%E5%90%8C%E9%A4%98  
  %\\ 叙述:
%同余\cite{WEBSITE:1}(英语:congruence modulo[1],符号:≡)是数论中的一种等价关系[2]。当两个整数除以同一个正整数,若得相同余数,则二整数同余。同余是抽象代数中的同余关系的原型[3]。最先引用同余的概念与“≡”符号者为德国数学家高斯。

Random citation \cite{WEBSITE:1} embeddeed in websit.

\subsection{裴蜀等式(Bézout's identity)}   
\subsection{辗转相除法}   
\subsection{扩展欧几里得算法}   
\subsection{模逆元(模反元素)}   
\subsection{欧拉定理}   
\subsection{欧拉函数}   
\subsection{费马小定理}   
\subsection{RSA算法}   
 

% 位于document中
\bibliography{rsa} 
\bibliographystyle{ieeetr}

\end{document}
