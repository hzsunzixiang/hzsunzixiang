
\documentclass{article}

\usepackage{amsthm}
\newtheorem{thm}{Theorem}[section]
\theoremstyle{definition}
\newtheorem{dfn}{Definition}[section]
\theoremstyle{remark}
\newtheorem{note}{Note}[section]
\theoremstyle{plain}
\newtheorem{lem}[thm]{Lemma}

\begin{document}

\section{example}
\begin{dfn}
A triangle is the figure formed by joining each pair
of three non collinear points by line segments.
\end{dfn}
\begin{note}
A triangle has three angles.
\end{note}
\begin{thm}
The sum of the angles of a triangle is $180ˆ\circ$.
\end{thm}
\begin{lem}
The sum of any two sides of a triangle is greater than or equal to the third.
\end{lem}

\end{document}
