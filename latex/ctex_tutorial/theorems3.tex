
\documentclass{article}



\usepackage{amsmath}

\usepackage{amsthm} %% 注意必须包含这个包 不然 通不过

\usepackage{amsfonts} % mathbb



\newcommand{\halmos}{\rule{1mm}{2.5mm}}
\renewcommand{\qedsymbol}{\halmos}

%There are some more predefined features in amsthm package. In all the different examples
%we have seen so far, the theorem number comes after the theorem name. Some prefer to
%have it the other way round as in
\swapnumbers
\theoremstyle{plain}
\newtheorem{mythm}{Theorem}[section]
% 修改这里
%\theoremstyle{mystyle}
%\theoremstyle{nonum}



\begin{document}


\section{example}

\begin{mythm}
The sum of the angles of a triangle is $180ˆ\circ$.
\end{mythm}
\begin{proof}
Let $\{p_1,p_2,\dotsc p_k\}$ be a finite set of primes. Define $n=p_1p_2\dotsm
p_k+1$. Then either $n$ itself is a prime or has a prime factor. Now $n$ is
neither equal to nor is divisible by any of the primes $p_1,p_2,\dotsc p_k$ so
that in either case, we get a prime different from $p_1,p_2,\dotsc p_k$. Thus
no finite set of primes can include all the primes.
\end{proof}


\begin{mythm}
The sum of the angles of a triangle is $180ˆ\circ$.
\end{mythm}
\begin{proof}
This follows easily from the equation
\begin{equation}
(x+y)ˆ2=xˆ2+yˆ2+2xy\tag*{\qed}
\end{equation}
\renewcommand{\qed}{}
\end{proof}

\begin{mythm}
The sum of the angles of a triangle is $180ˆ\circ$.
\end{mythm}
\begin{proof}
This follows easily from the equation
\begin{equation*}
(x+y)ˆ2=xˆ2+yˆ2+2xy\qed
\end{equation*}
\renewcommand{\qed}{}
\end{proof}

\end{document}
