
\documentclass{article}



\usepackage{amsmath}

\usepackage{amsthm} %% 注意必须包含这个包 不然 通不过

\usepackage{amsfonts} % mathbb


%\newtheoremstyle%
%{name}%
%{abovespace}%
%{belowspace}%
%{bodyfont}%
%{indent}%
%{headfont}%
%{headpunct}%
%{headspace}%
%{custom-head-spec}%

%\newtheoremstyle{mystyle}{}{}{\slshape}{}{\scshape}{.}{}{}
%\newtheoremstyle{mynewstyle}{12pt}{12pt}{\itshape}%
%{}{\sffamily}{:}{\newline}{}
%\theoremstyle{mystyle}
%\newtheorem{mythm}{Theorem}[section]


%\newtheoremstyle{plain}
%  {\topsep}   % ABOVESPACE
%  {\topsep}   % BELOWSPACE
%  {\itshape}  % BODYFONT
%  {0pt}       % INDENT (empty value is the same as 0pt)
%  {\bfseries} % HEADFONT
%  {.}         % HEADPUNCT
%  {5pt plus 1pt minus 1pt} % HEADSPACE
%  {}          % CUSTOM-HEAD-SPEC
%

% \newtheoremstyle{mystyle}{}{}{\slshape}{}{\scshape}{.}{ }{}

%\newtheoremstyle{mystyle}{12pt}{12pt}{\itshape}%
%{}{\sffamily}{:}{\newline}{}


%\newtheoremstyle{mystyle}{12pt}{12pt}{\itshape}%
%{}{\sffamily}{:}{2pt}{}

\newtheoremstyle{mystyle}{12pt}{12pt}{\itshape}%
{}{\sffamily}{......}{2pt}{}

\newtheoremstyle{nonum}{}{}{\itshape}{}{\bfseries}{.}{ }{#1 (\mdseries #3)}

\newtheoremstyle{newnonum}{}{}{\itshape}{}{\bfseries}{.}{ }%
{\thmname{#1}\thmnote{ (\mdseries #3)}}




\newtheoremstyle{citing}{}{}{\itshape}{}{\bfseries}{.}{ }{\thmnote{#3}}
\theoremstyle{citing}
\newtheorem{cit}{}

% 修改这里
%\theoremstyle{mystyle}
%\theoremstyle{nonum}
\theoremstyle{newnonum}


\newtheorem{mythm}{Theorem}[section]

\begin{document}


\section{example}

\begin{mythm}
The sum of the angles of a triangle is $180ˆ\circ$.
\end{mythm}


\begin{mythm}[Euclid]
The sum of the angles of a triangle is $180ˆ\circ$.
\end{mythm}

\begin{mythm}[Third Version]
If $G$ is a simply connected open subset of ${C}$, then for every closed
rectifiable curve $\gamma$ in $G$, we have
\begin{equation*}
\int_\gamma f=0.
\end{equation*}
\end{mythm}

\begin{mythm}Every open simply connected proper subset of $\mathbb{C}$ is analytically
homeomorphic to the open unit disk in $\mathbb{C}$.
\end{mythm}

\begin{cit}[Axiom 1 in \cite{eu}]
Things that are equal to the same thing are equal to one another.
\end{cit}



\end{document}
